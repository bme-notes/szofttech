\section{CMM - Capability Maturity Model}

Szintjei:
\begin{enumerate}
	\item Kezdetleges
		\begin{enumerate}
			\item A szoftver tervezése kaotikus.
			\item Néhány processz van csak definiálva
			\item A siker az egyes emberek küszködésein múlik
			\item A szoftver tervezés folyamata egy fekete doboz, nem látunk bele.
		\end{enumerate}

	\item Ismétlődő
		\begin{enumerate}
			\item Alapvető folyamatok megfigyelhetők: Költségvetés, időbeosztás, funkcionalitás
			\item Megismételhető, ujra felhasználható folyamatok
			\item A vevő elvárásai, és a munka folyamatok irányítva vannak.
			\item Már van project menedzsment - Ha rábökök egy sorra megtudjam mondani hogy miért került bele
			\item Lehetővé vált a költségek, az ütemterv és a funkcionalitás nyomonkövetése
		\end{enumerate}

	\item Definiált
		\begin{enumerate}
			\item van egy fejlesztési folyamatuk, ami megfelel valamilyen szabványnak

			\item a szoftverfolyamat tevékenységei már beintegrálódtak a szervezet szabványos szoftver-folyamatába

			\item minden projekt a szervezet szabványos fejlesztési és karbantartási folyamatának egy jóváhagyott, személyre szabott verzióját követi
		\end{enumerate}

	\item Menedzselt
		\begin{enumerate}
			\item Számszerűsítetten menedzselünk. Az kell hogy a folyamatokról Qvantitív számszerű értékeink legyenek.
			\item mérés van, a folyamat megjósolható
			\item a szoftverfolyamatról módszeresen adatokat gyűjtenek
			\item a vezetők képesek a folyamatok előrehaladásának és a problémáknak a mérésére
		\end{enumerate}

	\item Optimalizáló
		\begin{enumerate}
			\item Folyamat mérnökség.
			\item magát a folyamatot tervezhetjük termékfüggően
			\item ellenőrzött módon új és javított szoftverfejlesztési eszközöket próbálnak ki
			\item számszerű visszacsatolás segíti az állandó folyamatfejlesztést
		\end{enumerate}

\end{enumerate}
