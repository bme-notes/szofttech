%% BME-Notes jegyzethez header
%% Ha nem érted, mi történik itt, akkor inkább ne változtasd meg!
%% A fájlok fordításához XeLaTeX-et kell használni
\documentclass[]{article}
\usepackage{lmodern}
\usepackage{amssymb}
\usepackage{amsmath}
\usepackage{polyglossia}
\usepackage{listings}
\usepackage{tcolorbox}
\usepackage{etoolbox}
\usepackage{setspace}
\usepackage{framed}
\usepackage[a4paper,margin=2.5cm]{geometry}
\usepackage{fancyhdr}
\pagestyle{fancy}
\usepackage[hidelinks]{hyperref}
\definecolor{shadecolor}{HTML}{eeeeee} %Kindle-optimized color
\setcounter{secnumdepth}{0} %this automagically removes extra numbering toc and sections
\renewcommand{\contentsname}{Tartalomjegyzék}
\newtoks\cim
\newtoks\szerzo
\newtoks\segitettek
\newtoks\datum

\newcommand{\ujfejezet}[1]{\newpage \input{./fejezetek/#1.tex}}
\newenvironment{tetel}[1]{\begin{framed}\noindent\ignorespaces\textbf{\large Tétel: #1}\normalsize\\}{\end{framed}\ignorespacesafterend}
\newenvironment{definicio}[1]{\begin{shaded}\noindent\ignorespaces\textbf{\large Definíció: #1}\normalsize\\}{\end{shaded}\ignorespacesafterend}
\newenvironment{bizonyitas}[1]{\begin{leftbar}\noindent\ignorespaces\textbf{\large Bizonyítás: #1}\normalsize\\}{\end{leftbar}\ignorespacesafterend}


\title{\huge\textsc{\the\cim}}
\author{\the\szerzo}
\date{\the\datum}
